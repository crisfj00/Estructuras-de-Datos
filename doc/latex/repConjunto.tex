\hypertarget{repConjunto_invConjunto}{}\section{Invariante de la representacion}\label{repConjunto_invConjunto}
El invariante es rep.\{calorias,hidratos,proteinas,grasas,fibra\} $>$= 0\hypertarget{repConjunto_faConjunto}{}\section{Funcion de abstraccion}\label{repConjunto_faConjunto}
Un objeto valido {\itshape rep} del T\+DA ingrediente representa

Alimento ;Calorias ;Hidratos de Carb.;Proteinas;Grasas;Fibra;Tipo\hypertarget{repConjunto_invConjunto}{}\section{Invariante de la representacion}\label{repConjunto_invConjunto}
El invariante es rep.\+datos.\+size $>$0\hypertarget{repConjunto_faConjunto}{}\section{Funcion de abstraccion}\label{repConjunto_faConjunto}
Un objeto valido {\itshape rep} del T\+DA ingrediente se representa igual que {\itshape Ingrediente} ya que es un vector de estos\+:

Alimento;Calorias;Hidratos de Carb;Proteinas;Grasas;Fibra;Tipo\hypertarget{repConjunto_invConjunto}{}\section{Invariante de la representacion}\label{repConjunto_invConjunto}
Un invariante es rep.\+codigo tiene que ser diferente a cualquier otro (unico) Un invariante es rep.\+plato tiene que estar enre 1 y 3\hypertarget{repConjunto_faConjunto}{}\section{Funcion de abstraccion}\label{repConjunto_faConjunto}
Un objeto valido {\itshape rep} del T\+DA Receta representa

Codigo; Numero de plato (1-\/3); Nombre de la receta; Ingrediente 1 y la cantidad del ingrediente; ..... ; Ingrediente n y la cantidad del ingrediente 